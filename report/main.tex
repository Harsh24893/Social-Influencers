
%% bare_conf.tex
%% V1.3
%% 2007/01/11
%% by Michael Shell
%% See:
%% http://www.michaelshell.org/
%% for current contact information.
%%
%% This is a skeleton file demonstrating the use of IEEEtran.cls
%% (requires IEEEtran.cls version 1.7 or later) with an IEEE conference paper.
%%
%% Support sites:
%% http://www.michaelshell.org/tex/ieeetran/
%% http://www.ctan.org/tex-archive/macros/latex/contrib/IEEEtran/
%% and
%% http://www.ieee.org/

%%*************************************************************************
%% Legal Notice:
%% This code is offered as-is without any warranty either expressed or
%% implied; without even the implied warranty of MERCHANTABILITY or
%% FITNESS FOR A PARTICULAR PURPOSE! 
%% User assumes all risk.
%% In no event shall IEEE or any contributor to this code be liable for
%% any damages or losses, including, but not limited to, incidental,
%% consequential, or any other damages, resulting from the use or misuse
%% of any information contained here.
%%
%% All comments are the opinions of their respective authors and are not
%% necessarily endorsed by the IEEE.
%%
%% This work is distributed under the LaTeX Project Public License (LPPL)
%% ( http://www.latex-project.org/ ) version 1.3, and may be freely used,
%% distributed and modified. A copy of the LPPL, version 1.3, is included
%% in the base LaTeX documentation of all distributions of LaTeX released
%% 2003/12/01 or later.
%% Retain all contribution notices and credits.
%% ** Modified files should be clearly indicated as such, including  **
%% ** renaming them and changing author support contact information. **
%%
%% File list of work: IEEEtran.cls, IEEEtran_HOWTO.pdf, bare_adv.tex,
%%                    bare_conf.tex, bare_jrnl.tex, bare_jrnl_compsoc.tex
%%*************************************************************************

% *** Authors should verify (and, if needed, correct) their LaTeX system  ***
% *** with the testflow diagnostic prior to trusting their LaTeX platform ***
% *** with production work. IEEE's font choices can trigger bugs that do  ***
% *** not appear when using other class files.                            ***
% The testflow support page is at:
% http://www.michaelshell.org/tex/testflow/



% Note that the a4paper option is mainly intended so that authors in
% countries using A4 can easily print to A4 and see how their papers will
% look in print - the typesetting of the document will not typically be
% affected with changes in paper size (but the bottom and side margins will).
% Use the testflow package mentioned above to verify correct handling of
% both paper sizes by the user's LaTeX system.
%
% Also note that the "draftcls" or "draftclsnofoot", not "draft", option
% should be used if it is desired that the figures are to be displayed in
% draft mode.
%
\documentclass[conference]{IEEEtran}
\usepackage{blindtext, graphicx}
% Add the compsoc option for Computer Society conferences.
%
% If IEEEtran.cls has not been installed into the LaTeX system files,
% manually specify the path to it like:
% \documentclass[conference]{../sty/IEEEtran}
% Some very useful LaTeX packages include:
% (uncomment the ones you want to load)


% *** MISC UTILITY PACKAGES ***
%
%\usepackage{ifpdf}
% Heiko Oberdiek's ifpdf.sty is very useful if you need conditional
% compilation based on whether the output is pdf or dvi.
% usage:
% \ifpdf
%   % pdf code
% \else
%   % dvi code
% \fi
% The latest version of ifpdf.sty can be obtained from:
% http://www.ctan.org/tex-archive/macros/latex/contrib/oberdiek/
% Also, note that IEEEtran.cls V1.7 and later provides a builtin
% \ifCLASSINFOpdf conditional that works the same way.
% When switching from latex to pdflatex and vice-versa, the compiler may
% have to be run twice to clear warning/error messages.






% *** CITATION PACKAGES ***
%
%\usepackage{cite}
% cite.sty was written by Donald Arseneau
% V1.6 and later of IEEEtran pre-defines the format of the cite.sty package
% \cite{} output to follow that of IEEE. Loading the cite package will
% result in citation numbers being automatically sorted and properly
% "compressed/ranged". e.g., [1], [9], [2], [7], [5], [6] without using
% cite.sty will become [1], [2], [5]--[7], [9] using cite.sty. cite.sty's
% \cite will automatically add leading space, if needed. Use cite.sty's
% noadjust option (cite.sty V3.8 and later) if you want to turn this off.
% cite.sty is already installed on most LaTeX systems. Be sure and use
% version 4.0 (2003-05-27) and later if using hyperref.sty. cite.sty does
% not currently provide for hyperlinked citations.
% The latest version can be obtained at:
% http://www.ctan.org/tex-archive/macros/latex/contrib/cite/
% The documentation is contained in the cite.sty file itself.






% *** GRAPHICS RELATED PACKAGES ***
%
\ifCLASSINFOpdf
  % \usepackage[pdftex]{graphicx}
  % declare the path(s) where your graphic files are
  % \graphicspath{{../pdf/}{../jpeg/}}
  % and their extensions so you won't have to specify these with
  % every instance of \includegraphics
  % \DeclareGraphicsExtensions{.pdf,.jpeg,.png}
\else
  % or other class option (dvipsone, dvipdf, if not using dvips). graphicx
  % will default to the driver specified in the system graphics.cfg if no
  % driver is specified.
  % \usepackage[dvips]{graphicx}
  % declare the path(s) where your graphic files are
  % \graphicspath{{../eps/}}
  % and their extensions so you won't have to specify these with
  % every instance of \includegraphics
  % \DeclareGraphicsExtensions{.eps}
\fi
% graphicx was written by David Carlisle and Sebastian Rahtz. It is
% required if you want graphics, photos, etc. graphicx.sty is already
% installed on most LaTeX systems. The latest version and documentation can
% be obtained at: 
% http://www.ctan.org/tex-archive/macros/latex/required/graphics/
% Another good source of documentation is "Using Imported Graphics in
% LaTeX2e" by Keith Reckdahl which can be found as epslatex.ps or
% epslatex.pdf at: http://www.ctan.org/tex-archive/info/
%
% latex, and pdflatex in dvi mode, support graphics in encapsulated
% postscript (.eps) format. pdflatex in pdf mode supports graphics
% in .pdf, .jpeg, .png and .mps (metapost) formats. Users should ensure
% that all non-photo figures use a vector format (.eps, .pdf, .mps) and
% not a bitmapped formats (.jpeg, .png). IEEE frowns on bitmapped formats
% which can result in "jaggedy"/blurry rendering of lines and letters as
% well as large increases in file sizes.
%
% You can find documentation about the pdfTeX application at:
% http://www.tug.org/applications/pdftex





% *** MATH PACKAGES ***
%
%\usepackage[cmex10]{amsmath}
% A popular package from the American Mathematical Society that provides
% many useful and powerful commands for dealing with mathematics. If using
% it, be sure to load this package with the cmex10 option to ensure that
% only type 1 fonts will utilized at all point sizes. Without this option,
% it is possible that some math symbols, particularly those within
% footnotes, will be rendered in bitmap form which will result in a
% document that can not be IEEE Xplore compliant!
%
% Also, note that the amsmath package sets \interdisplaylinepenalty to 10000
% thus preventing page breaks from occurring within multiline equations. Use:
%\interdisplaylinepenalty=2500
% after loading amsmath to restore such page breaks as IEEEtran.cls normally
% does. amsmath.sty is already installed on most LaTeX systems. The latest
% version and documentation can be obtained at:
% http://www.ctan.org/tex-archive/macros/latex/required/amslatex/math/





% *** SPECIALIZED LIST PACKAGES ***
%
%\usepackage{algorithmic}
% algorithmic.sty was written by Peter Williams and Rogerio Brito.
% This package provides an algorithmic environment fo describing algorithms.
% You can use the algorithmic environment in-text or within a figure
% environment to provide for a floating algorithm. Do NOT use the algorithm
% floating environment provided by algorithm.sty (by the same authors) or
% algorithm2e.sty (by Christophe Fiorio) as IEEE does not use dedicated
% algorithm float types and packages that provide these will not provide
% correct IEEE style captions. The latest version and documentation of
% algorithmic.sty can be obtained at:
% http://www.ctan.org/tex-archive/macros/latex/contrib/algorithms/
% There is also a support site at:
% http://algorithms.berlios.de/index.html
% Also of interest may be the (relatively newer and more customizable)
% algorithmicx.sty package by Szasz Janos:
% http://www.ctan.org/tex-archive/macros/latex/contrib/algorithmicx/




% *** ALIGNMENT PACKAGES ***
%
%\usepackage{array}
% Frank Mittelbach's and David Carlisle's array.sty patches and improves
% the standard LaTeX2e array and tabular environments to provide better
% appearance and additional user controls. As the default LaTeX2e table
% generation code is lacking to the point of almost being broken with
% respect to the quality of the end results, all users are strongly
% advised to use an enhanced (at the very least that provided by array.sty)
% set of table tools. array.sty is already installed on most systems. The
% latest version and documentation can be obtained at:
% http://www.ctan.org/tex-archive/macros/latex/required/tools/


%\usepackage{mdwmath}
%\usepackage{mdwtab}
% Also highly recommended is Mark Wooding's extremely powerful MDW tools,
% especially mdwmath.sty and mdwtab.sty which are used to format equations
% and tables, respectively. The MDWtools set is already installed on most
% LaTeX systems. The lastest version and documentation is available at:
% http://www.ctan.org/tex-archive/macros/latex/contrib/mdwtools/


% IEEEtran contains the IEEEeqnarray family of commands that can be used to
% generate multiline equations as well as matrices, tables, etc., of high
% quality.


%\usepackage{eqparbox}
% Also of notable interest is Scott Pakin's eqparbox package for creating
% (automatically sized) equal width boxes - aka "natural width parboxes".
% Available at:
% http://www.ctan.org/tex-archive/macros/latex/contrib/eqparbox/





% *** SUBFIGURE PACKAGES ***
%\usepackage[tight,footnotesize]{subfigure}
% subfigure.sty was written by Steven Douglas Cochran. This package makes it
% easy to put subfigures in your figures. e.g., "Figure 1a and 1b". For IEEE
% work, it is a good idea to load it with the tight package option to reduce
% the amount of white space around the subfigures. subfigure.sty is already
% installed on most LaTeX systems. The latest version and documentation can
% be obtained at:
% http://www.ctan.org/tex-archive/obsolete/macros/latex/contrib/subfigure/
% subfigure.sty has been superceeded by subfig.sty.



%\usepackage[caption=false]{caption}
%\usepackage[font=footnotesize]{subfig}
% subfig.sty, also written by Steven Douglas Cochran, is the modern
% replacement for subfigure.sty. However, subfig.sty requires and
% automatically loads Axel Sommerfeldt's caption.sty which will override
% IEEEtran.cls handling of captions and this will result in nonIEEE style
% figure/table captions. To prevent this problem, be sure and preload
% caption.sty with its "caption=false" package option. This is will preserve
% IEEEtran.cls handing of captions. Version 1.3 (2005/06/28) and later 
% (recommended due to many improvements over 1.2) of subfig.sty supports
% the caption=false option directly:
%\usepackage[caption=false,font=footnotesize]{subfig}
%
% The latest version and documentation can be obtained at:
% http://www.ctan.org/tex-archive/macros/latex/contrib/subfig/
% The latest version and documentation of caption.sty can be obtained at:
% http://www.ctan.org/tex-archive/macros/latex/contrib/caption/




% *** FLOAT PACKAGES ***
%
%\usepackage{fixltx2e}
% fixltx2e, the successor to the earlier fix2col.sty, was written by
% Frank Mittelbach and David Carlisle. This package corrects a few problems
% in the LaTeX2e kernel, the most notable of which is that in current
% LaTeX2e releases, the ordering of single and double column floats is not
% guaranteed to be preserved. Thus, an unpatched LaTeX2e can allow a
% single column figure to be placed prior to an earlier double column
% figure. The latest version and documentation can be found at:
% http://www.ctan.org/tex-archive/macros/latex/base/



%\usepackage{stfloats}
% stfloats.sty was written by Sigitas Tolusis. This package gives LaTeX2e
% the ability to do double column floats at the bottom of the page as well
% as the top. (e.g., "\begin{figure*}[!b]" is not normally possible in
% LaTeX2e). It also provides a command:
%\fnbelowfloat
% to enable the placement of footnotes below bottom floats (the standard
% LaTeX2e kernel puts them above bottom floats). This is an invasive package
% which rewrites many portions of the LaTeX2e float routines. It may not work
% with other packages that modify the LaTeX2e float routines. The latest
% version and documentation can be obtained at:
% http://www.ctan.org/tex-archive/macros/latex/contrib/sttools/
% Documentation is contained in the stfloats.sty comments as well as in the
% presfull.pdf file. Do not use the stfloats baselinefloat ability as IEEE
% does not allow \baselineskip to stretch. Authors submitting work to the
% IEEE should note that IEEE rarely uses double column equations and
% that authors should try to avoid such use. Do not be tempted to use the
% cuted.sty or midfloat.sty packages (also by Sigitas Tolusis) as IEEE does
% not format its papers in such ways.


% *** PDF, URL AND HYPERLINK PACKAGES ***
%
%\usepackage{url}
% url.sty was written by Donald Arseneau. It provides better support for
% handling and breaking URLs. url.sty is already installed on most LaTeX
% systems. The latest version can be obtained at:
% http://www.ctan.org/tex-archive/macros/latex/contrib/misc/
% Read the url.sty source comments for usage information. Basically,
% \url{my_url_here}.





% *** Do not adjust lengths that control margins, column widths, etc. ***
% *** Do not use packages that alter fonts (such as pslatex).         ***
% There should be no need to do such things with IEEEtran.cls V1.6 and later.
% (Unless specifically asked to do so by the journal or conference you plan
% to submit to, of course. )


% correct bad hyphenation here
\usepackage[english]{babel} % English language/hyphenation
\usepackage{array}
\usepackage{multirow}
\usepackage{amsmath}
\numberwithin{equation}{section}
\numberwithin{figure}{section}
\numberwithin{table}{section}
\usepackage{caption}
\usepackage{listings}
\usepackage{color}
 
\definecolor{codegreen}{rgb}{0,0.6,0}
\definecolor{codegray}{rgb}{0.5,0.5,0.5}
\definecolor{codepurple}{rgb}{0.58,0,0.82}
\definecolor{backcolour}{rgb}{0.95,0.95,0.92}
 
\lstdefinestyle{mystyle}{
    backgroundcolor=\color{backcolour},
    commentstyle=\color{codegreen},
    keywordstyle=\color{magenta},
    numberstyle=\tiny\color{codegray},
    stringstyle=\color{codepurple},
    basicstyle=\footnotesize,
    breakatwhitespace=false,         
    breaklines=true,                 
    captionpos=b,                    
    keepspaces=true,                 
    numbers=left,                    
    numbersep=5pt,                  
    showspaces=false,                
    showstringspaces=false,
    showtabs=false,                  
    tabsize=2
}
\lstset{style=mystyle}
\begin{document}

%
% paper title
% can use linebreaks \\ within to get better formatting as desired
\title{A Comparative study of different Machine Learning techniques on Kaggle Twitter Social Influencers Data Set}
\author{\IEEEauthorblockN{Arora, Pragya\IEEEauthorrefmark{1},
Ghai, Piyush\IEEEauthorrefmark{2}, Gupta, Harsh\IEEEauthorrefmark{3} and
Ramkrishnan, Navnith\IEEEauthorrefmark{4}}
\IEEEauthorblockA{Department of Computer Science \& Engineering,
The Ohio State University\\
Columbus, OH 43202\\
Email: \IEEEauthorrefmark{1}arora.170@osu.edu,
\IEEEauthorrefmark{2}ghai.8@osu.edu,
\IEEEauthorrefmark{3}gupta.749@osu.edu,
\IEEEauthorrefmark{4}ramkrishnan.1@osu.edu}}
\maketitle


\begin{abstract}
%\boldmath
Social media like twitter has given advent to growing importance to understand the behavior of these networks. More importantly these graph networks shows different patterns which are interesting to analyze. Some nodes becomes almost a sink where as some node acts like a source of outgoing links. Understanding these patterns will help in various applications and specially in experiments like Network A/B Testing where we need to introduce some new features to a treatment group restricting the control group. In out project we intended to experiment and analyze on one such pattern where we try to identify which node is more influential when compared to another node. To understand such a behavior we used a Twitter network data set which had 11 features for each node and we applied machine learning techniques to understanding the underlying features to classify the task at hand.
 
\end{abstract}
% IEEEtran.cls defaults to using nonbold math in the Abstract.
% This preserves the distinction between vectors and scalars. However,
% if the journal you are submitting to favors bold math in the abstract,
% then you can use LaTeX's standard command \boldmath at the very start
% of the abstract to achieve this. Many IEEE journals frown on math
% in the abstract anyway.

% Note that keywords are not normally used for peerreview papers.
\section{Introduction}
The growing importance of Social Networks has given an impetus to understand more about its implications because networks like these can influence millions of people. Contrary to paper or television media, social media is far more reaching to the audiences across the world and information exchange has never seen such a revolution. The need of the time is to have an The ability to understand the nuances of these networks because with more connectivity it becomes crucial to understand the behavior of different nodes in the network.

Some nodes/users are more important than others as there behavior affects a larger sets of nodes in the graph. When a user of such a graph is using the network she/he may find some activity shared by some of the other nodes more striking than others. A lot of psychological and cognitive science study shows that ones life is greatly impacted by her/his immediately surrounding and their influence. As social network has become a part of everyone's life, it has become important which are the nodes which has more impact on others i.e which are influential. One application of such a study is to predict what is the general mood of the crowd during election by analyzing the social network handle of all the candidates who stood for it. 

In this project, we address a targeted challenge of this problem presented in a Kaggle competition from 2013 , in which influence on Twitter is predicted from a set of features derived from user activity on the network, such as follower count, retweets received, etc. Rather than identifying influencers in the larger network, we are asked to simply identify which of a pair of users, A and B, is more influential, based on information about their respective activity in the network. This is a binary-classification problem. To find an optimal prediction method, we applied pre-processing methods including feature transformations and dimensionality reduction via principal component analysis (PCA). We used four different machinelearning algorithms—Logistic Regression, Support Vector Machines (SVM), Neural Network, and Gradient Boosting—to model the data. We also tried varying some of the decision boundary options for the SVM by using different kernels and neural network classifiers by varying the activation functions on the hidden layer. Our approach includes a few methods that we did not see in other attempts at this problem, specifically a binary feature transformation, the application of principal component analysis, and gradient boosting. We found the most effective pre-processing method to be a logarithmic transformation of the data, while the most successful model was built with gradient boosting, which achieved an 0.869 AUC using a logarithm transform. Outside of gradient boosting, the model configurations which produced the best results used linear-like decision boundaries.
 

\section{DATASET}\label{sec:page-layout}

This section expands on the data set which we used to try the different models on. 


\subsection{Original Dataset}\label{sec:formatting}
The data set which we used came from a Kaggle contest. The data set had a 
\begin{itemize}
  \item train.csv
  \item test.csv
  \item sample\_predicitions.csv
\end{itemize}

The train.csv file contains data points where each data point is a pair wise features for two users along with a binary value which reflects which user is more influential. If the value is 0 the first user is more influential and vice-verse. Each user has 11 unique features which are extracted from their twitter profile and recorded in numeric form. The 11 features extracted includes:

\begin{itemize}
  \item Follwers
  \item Follwing
  \item Listed
  \item Mentions received
  \item Re-tweets received
  \item Mentions sent
  \item Re-tweets sent
  \item Posts
  \item Network Feature 1
  \item Network Feature 2
  \item Network Feature 2
\end{itemize}

Similarly we have features for two users in each data item in the testing file but the testing file has no class label. There are 3000 training samples and 2500 testing samples in our data set. Given a test sample, our job is to predict which individual in this test sample is more influential.
 


%% Notice that paragraph breaks are done with a blank line between paragraphs


\subsection{Data Preprocessing
}\label{sec:formatting-text}

Since the data is a numerical data, we use some pre-processing techniques on it. The dataset consists of 22 attributes which is essentially 11 attributes for each user in contention. Since we have to compare who is more influential in the network, we subtract related attributes for each user, to reduce the dimensionality to 11. The following is the transformation applied : 
\begin{align*}
a\textsubscript{j} = x\textsubscript{j} - x\textsubscript{j+11}, j = 1,2,3 .... , 11. 
\end{align*}
Since the attributes are of a different scale, we also apply a logarithmic transformation on the data. The following is the transformation applied :
\begin{lstlisting}[language=Python, caption = Log Transformation ]
def transform_features(x):
    return np.log(1 + x)
\end{lstlisting}
All of these transformations are applied to all the models used.
\section{SYSTEM MODELS
}\label{sec:fig-tables}


\subsection{Baseline}\label{sec:cap-num}
For baseline, we tried a dumb baseline model, where we classified an influencer based on the number of followers, i.e. if X has more followers than Y, then X is more influential. Using this dumb baseline, we get an accuracy of about \textbf{70.2\%}. This result shows
that the number of followers is a strong indicator
of the influencers. However, 70.2\% is not a very satisfying result and hence we use it only as a benchmark. We know further experiment with more models and a bit more pre-processing.


\subsection{Logistic Regression}\label{sec:colour-illustrations}
First, we preprocess the original dataset using the
previous method. After preprocessing the original
dataset, different attributes have different ranges
which vary a lot. Thus the first thing to do is to
handle the data to make it more uniform and easy
to deal with. So in order to achieve this, we have a
normalization on the dataset. For each feature, we
do the following normalization
z¯
(i)
j =
z
(i)
q
j
Pm
i=1 z
(i)2
j
, z = 1, 2, ..., n (4)
where m is the number of training examples and n = 11 is the number of features. Because our job is to predict who is more influential given two persons, the number of followers, as we know, plays a significant rule in the prediction.
A person with more followers than the other is more likely to be judged influential by users. So we will
multiply the normalization of the first feature, i.e. number of followers, by a positive constant factor
which is greater than 1 to make it more influential than other features on the prediction. For this logistic regression problem, we have two choices— gradient ascent and Newton’s method to achieve the parameter θ. Since the number of
features n = 11 is not very large, so it will not take much time to compute the inverse of a n×n matrix.
Thus Newton’s method should converge faster in this problem and we choose Newton’s method as our implementation. The update rule of the Newton’s method is:
θ := θ − H
−1∇θl(θ) (5)
where H ∈ R
n×n
is the Hessian matrix and Hjk =
−
Pm
i=1 hθ(¯z
(i)
)(1 − hθ(¯z
(i)
))¯z
(i)
j
z¯
(i)
k
, ∇θl(θ) =
Z¯(~y − hθ(Z¯)).
Furthermore, we will add cross validation and feature selection to this model, which will be discussed in details in section IV. We write our own code to implement the Newton’s method.


\subsection{SVM}\label{sec:colour-illustrations}

We applied the technique of SVM's with a linear kernel and the results were on par with those of Logistic Regression with an AUC of x. Since the original feature set consisted of 11 features, intuitively mapping them to a higher dimensional space wouldn't improve performance. Thus we preferred the linear kernel over the non-linear kernels since it is a simpler model and hence less prone to over-fitting. 

Moreover, we add cross validation and feature
selection to this model, which refers to section IV.

\subsection{Gaussian Na{\"i}ve Bayes}\label{sec:colour-illustrations}
Logistic regression and SVM are discriminative learning algorithms. Gaussian Na{\"i}ve Bayes, is a generative model. The distribution of the original data is not Gaussian, so we tried Z-Score normalization in order to make it as a Gaussian distribution. In Z-Score, we do the following for every value : 
\begin{align*}
z = \frac{x - \mu}{\sigma}
\end{align*}
This converts the data into 0 mean and deviation 1. We then used Gaussian Na{\"i}ve Bayes from \textit{sklearn} package.

\subsection{Neural Network}\label{sec:colour-illustrations}
We wanted to explore and see how the date set behave in a non-linear environment. The idea was to compare the traditional techniques alongside some state of the art machine learning models which have gained a lot of importance today. Hence we implemented a basic Neural Network model in Python to see how it behaves in this classification task. Our model was customizable which means we could add or remove any number of hidden layers with any number of units in each of them. To our surprise the model worked well but didn't outperform the traditional model. The intuition behind this is that the data set and the features we have are linearly separable. 

\section{Tuning the Models}
\subsection{Feature Selection}
Since our dataset has 11 features, it is quite possible that there are some features which are not quite useful. We used forward selection algorithm to zero in on the best features in order to improve the test accuracy. We used Logistic Regression model in order to select the best accuracies for feature selection. The best features are as follows :
\begin{itemize}
\item Follower Count
\item Listed\_Count
\item Retweets Received
\item Network Feature 1
\item Network Feature 2
\end{itemize}

\subsection{Cross Validation}
We use held out cross validation, where we partition 80\% of the data as the training set and 20\% of the data as the test set. 

\section{Results \& Discussions}
\subsection{Evaluation Criteria}



\subsection{Logistic Regression \& SVM}
In this section, we apply cross validation and
feature selection to the linear models, i.e. Logistic
Regression and SVM. The following figures compare
the performance among different models.
In Fig. 2, we use different cross validation options
on linear models, such as hold-out and k-fold,
Fig. 2. Test accuracy vs. cross validation options for linear models
and compare their test accuracies. It is shown that
cross validation can improve the test accuracy of
SVM, while having no strong effect on Logistic
Regression. What’s more, the performance of SVM
is better than Logistic Regression using cross validation.
Fig. 3. Test accuracy vs. discretization parameter for linear models
Fig. 3 shows the test accuracies of SVM and Logistic
Regression with different numbers of features
selected. As is shown above, the performance of
SVM is better, compared with logistic regression.
The best performance of SVM is achieved when
4 features are selected, which shows that some
features are weak indicators. After feature selection,
we sort the features from the most relevant to the
least relevant as follows:
SVM 9 6 7 10 8 5 3 2 1 4 11
LR 3 6 8 2 9 5 4 7 1 11 10
where the corresponding feature name refers to section
II-A. From this table, we can see that although
the sort results shown above are quite different for
these two methods, their test accuracies are very
close. Using linear models, the best accuracy we
can achieve is 76.00% (highest test accuracy in Fig.
2 and Fig. 3).
\subsection{Naive Bayes}
In this model, we compare the performances of
different discretization parameters. Here we assume
that all features have the same number of classes
(the coordinate ascent based algorithm takes too
much time). We also mentioned in section III-D that
logarithm function is used to preprocess the data.
2 3 4 5 6 7 8 9 10
0.62
0.64
0.66
0.68
0.7
0.72
0.74
0.76
0.78
0.8
\# Classes of each attribute
Accuracy
With Logarithm
Without Logarithm
Fig. 4. Test accuracies vs. number of classes of each feature
In Fig. 4, we also compare the performance with
logarithm preprocessing and the one without logarithm
processing. We can see that logarithm processing
helps increase the test accuracy and when the
number of classes of each feature increases, their
performances approaches. In addition, it turns out
that different discretization parameters don’t have
much impact on test accuracy. In this case, the best
test accuracy is 76.48%.
\subsection{Neural Network}
In this model, we use 50 neurons in the hidden
layer. We also apply hold-out cross validation with
30% as the validation set. The ROC (Receiver
Operating Characteristic) curves are shown in Fig.
5. After training the two-layer network, we get the
test accuracy 79.04%, which corresponds to the
area under the test ROC curve.
0 0.2 0.4 0.6 0.8 1
0
0.1
0.2
0.3
0.4
0.5
0.6
0.7
0.8
0.9
1
False Positive Rate
True Positive Rate
Training ROC
0 0.2 0.4 0.6 0.8 1
0
0.1
0.2
0.3
0.4
0.5
0.6
0.7
0.8
0.9
1
False Positive Rate
True Positive Rate
Validation ROC
0 0.2 0.4 0.6 0.8 1
0
0.1
0.2
0.3
0.4
0.5
0.6
0.7
0.8
0.9
1
False Positive Rate
True Positive Rate
Test ROC
0 0.2 0.4 0.6 0.8 1
0
0.1
0.2
0.3
0.4
0.5
0.6
0.7
0.8
0.9
1
False Positive Rate
True Positive Rate
All ROC
Fig. 5. Receiver operating characteristic

\section{CONCLUSION \& FUTURE WORKS}
From the above results, we can conclude that the
best accuracy we can achieve is about 76% using
linear models, which is not much better than our
benchmark 70.2%. This suggests that the testing
examples might not be linearly separable. The test
accuracy of Naive Bayes is close to linear models.
However, if we can apply the coordinate ascent
based algorithm, we may achieve much better performance,
which is a good choice for future works.
Furthermore, the accuracy of nonlinear models such
as Neural Network is better than linear models,
although it’s not as good as we expected. Moreover,
there might be data corruptions since sometimes
human judgement can be highly biased. In order to
achieve better performance, we can either try more
nonlinear models or use decision trees to improve
it.

% (used to reserve space for the reference number labels box)
\begin{thebibliography}{1}
\bibitem{iseardataset}
The ISEAR Dataset is available from : http://emotion-research.net/toolbox/toolboxdatabase.2006-10-13.2581092615

\bibitem{sentiword}
SentiWordNet is a lexical resource for opinion mining. More about it can be found at : http://sentiwordnet.isti.cnr.it/

\bibitem{esna}
S. Mostafa Al Masum, H. Prendinger and M. Ishizuka, "Emotion Sensitive News Agent: An Approach Towards User Centric Emotion Sensing from the News," Web Intelligence, IEEE/WIC/ACM International Conference on, Fremont, CA, 2007, pp. 614-620.
doi: 10.1109/WI.2007.124

\bibitem{word2vec}
Word2Vec tool provides an efficient implementation of the continuous bag-of-words and skip-gram architectures for computing vector representations of words. These representations can be subsequently used in many natural language processing applications and for further research. https://code.google.com/archive/p/word2vec/

\bibitem{glove}
GloVe is an unsupervised learning algorithm for obtaining vector representations for words. Training is performed on aggregated global word-word co-occurrence statistics from a corpus, and the resulting representations showcase interesting linear substructures of the word vector space. https://nlp.stanford.edu/projects/glove/

\bibitem{lstm}
The following is a good link about understanding LSTMs : http://colah.github.io/posts/2015-08-Understanding-LSTMs/

\bibitem{lstm2}
A Keras implementation of using pre-trained word embeddings.
https://blog.keras.io/using-pre-trained-word-embeddings-in-a-keras-model.html
\bibitem{kimyoon}
Yoon Kim, Convulational Neural Networks for Sentence Classification, http://arxiv.org/abs/1408.5882

\bibitem{wildmlcnn}
An implementation for CNNs for Sentence Classification can be found at : http://www.wildml.com/2015/12/implementing-a-cnn-for-text-classification-in-tensorflow/

\end{thebibliography}

% biography section
% 
% If you have an EPS/PDF photo (graphicx package needed) extra braces are
% needed around the contents of the optional argument to biography to prevent
% the LaTeX parser from getting confused when it sees the complicated
% \includegraphics command within an optional argument. (You could create
% your own custom macro containing the \includegraphics command to make things
% simpler here.)
%\begin{biography}[{\includegraphics[width=1in,height=1.25in,clip,keepaspectratio]{mshell}}]{Michael Shell}
% or if you just want to reserve a space for a photo:

% You can push biographies down or up by placing
% a \vfill before or after them. The appropriate
% use of \vfill depends on what kind of text is
% on the last page and whether or not the columns
% are being equalized.

%\vfill

% Can be used to pull up biographies so that the bottom of the last one
% is flush with the other column.
%\enlargethispage{-5in}


% that's all folks
\end{document}
